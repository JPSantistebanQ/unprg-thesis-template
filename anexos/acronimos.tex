% Lista de acrónimos (se ordenan por orden alfabético automáticamente)

% La forma de definir un acrónimo es la siguiente:
% \newacronyn{id}{siglas}{descripción}
% Donde:
% 	'id' es como vas a llamarlo desde el documento.
%	'siglas' son las siglas del acrónimo.
%	'descripción' es el texto que representan las siglas.
%
% Para usarlo en el documento tienes 4 formas:
% \gls{id} - Añade el acrónimo en su forma larga y con las siglas si es la primera vez que se utiliza, el resto de veces solo añade las siglas. (No utilices este en títulos de capítulos o secciones).
% \glsentryshort{id} - Añade solo las siglas de la id
% \glsentrylong{id} - Añade solo la descripción de la id
% \glsentryfull{id} - Añade tanto  la descripción como las siglas

%\newacronym{tfg}{TFG}{Trabajo Final de Grado}
%\newacronym{tfm}{TFM}{Trabajo Final de Máster}

\newacronym{iso}{ISO}{La Organización Internacional de Normalización}
\newacronym{ui}{UI}{User Interface}
\newacronym{tupa}{TUPA}{Texto Único de Procedimientos Administrativos}
\newacronym{uml}{UML}{Lenguaje Unificado de Modelado}

% A las descripciones se les agrega el punto final automaticamente

% ACRONIMOS
\newglossaryentry{javascript}{
	name=JavaScript,
	description={Lenguaje de programación orientado a objetos, diseñado para el desarrollo de aplicaciones cliente/servidor a través de Internet}
}

\newglossaryentry{nodejs}{
	name=NodeJS,
	description={Entorno de Ejecución Multiplataforma que permite ejecutar \gls{javascript} en navegadores, servidores, PC's, smartphones o en cualquier sitio donde se consiga portar un motor V8 de Google}
}

\newglossaryentry{framework}{
	name=Framework,
	description={Es un conjunto integrado de herramientas que facilitan un desarrollo software}
}