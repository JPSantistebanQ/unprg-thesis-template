% !TeX program = xelatex
% !TeX TXS-program:compile = txs:///xelatex/[--shell-escape]
%%%%%%%%%%%%%%%%%%%%%%%%%%%%%%%%%%%%%%%%%%%%%%%%%%%%%%%%%%%%%%%%%%%%%%%%
% Plantilla TFG/TFM
% Escuela Politécnica Superior de la Universidad de Alicante
% Realizado por: Jose Manuel Requena Plens
% Contacto: info@jmrplens.com / Telegram:@jmrplens
%%%%%%%%%%%%%%%%%%%%%%%%%%%%%%%%%%%%%%%%%%%%%%%%%%%%%%%%%%%%%%%%%%%%%%%%

% Elige si deseas optimizar la ejecución del proyecto almacenando las figuras generadas con TikZ y PGF en una carpeta (archivos/figuras-procesadas).
% 1 - Si, 2 - No
\def\OptimizaTikZ{1}

% Archivo .TEX que incluye todas las configuraciones del documento y los paquetes. Añade todo aquello que necesites utilizar en el documento en este archivo.
% En él se encuentra la configuración de los márgenes, establecidos según las directrices de estilo de la EPS.
\input{include/configuracioninicial}

%%%%%%%%%%%%%%%%%%%%%%%%%%%%%%%%%%%%%%%%%%%%%%%%%%%%%%%%%%%%%%%%%%%%%%
% INFORMACIÓN DEL TFG
% Comentar lo que NO se desee añadir y sustituir con la información correcta.
%%%%%%%%%%%%%%%%%%%%%%%%%%%%%%%%%%%%%%%%%%%%%%%%%%%%%%%%%%%%%%%%%%%%%%
% Título y subtítulo
\newcommand{\titulo}{Aplicación Móvil y Plataforma Web para la Gestión de Licencias de la Sub Gerencia de Control Urbano y Acondicionamiento Territorial de la Municipalidad Provincial de Lambayeque}
\newcommand{\subtitulo}{utilizando los frameworks React Native y ReactJS}
% Datos del autor
\newcommand{\miNombre}{
	Apellidos y nombres del autor}
\newcommand{\miEmail}{}
% Datos del tutor/es
\newcommand{\miTutor}{Mg. Ing. Apellidos y nombres del asesor}
\newcommand{\departamentoTutor}{}
% Datos de la facultada y universidad
\newcommand{\miFacultad}{Facultad de Ingeniería Civil, Sistemas y Arquitectura}
\newcommand{\miFacultadCorto}{FICSA}
\newcommand{\miEscuela}{Escuela Profesional de Ingeniería de Sistemas}
\newcommand{\miEscuelaCorto}{EPIS}
\newcommand{\miUniversidad}{\protect{Universidad Nacional Pedro Ruiz Gallo}}
\newcommand{\miUniversidadTwoLine}{\protect{Universidad Nacional\\Pedro Ruiz Gallo}}
\newcommand{\miUbicacion}{Lambayeque}
\newcommand{\miPais}{Perú}

%%%%%%%%%%%%%%%%%%%%%%%%%%%%%%%%%%%%%%%%%%%%%%%%%%%%%%%%%%%%%%%%%%%%%%
% INDICA TU TITULACIÓN
% ID	GRADO -------------------------------------------------
% 1		Ingeniería en Imagen y Sonido en Telecomunicación
% 2		Ingeniería Civil
% 3		Ingeniería Química
% 4		Ingeniería Informática
% 5		Ingeniería Multimedia
% 6		Arquitectura Técnica
% 7		Arquitectura
% 8		Robótica
% %		%%%%%%%%%%%%
% ID	MÁSTER ------------------------------------------------
% A		Telecomunicación
% B		Caminos, Canales y Puertos
% C		Gestión en la Edificación
% D		Desarrollo Web
% E		Materiales, Agua, Terreno
% F		Informática
% G 	Automática y Robótica
% H		Prevención de riesgos laborales
% I		Gestión Sostenible Agua
% J		Desarrollo Aplicaciones Móviles
% K		Ingeniería Química
% L		Ciberseguridad
%%%%%%%%%%%%%%%%%%%%%%%%%%%%%%%%%%%%%%%%%%%%%%%%%%%%%%%%%%%%%%%%%%%%%%%%%
%!!!!!!!!!!!!!!!!!!!!!!!!!!!!!!!!!!!!!!!!!!!!!!!!!!!!!!!!!!!!!!!!!!!!!%%%
																		%
\def\IDtitulo{9} % INTRODUCE LA ID DE TU TITULACIÓN						%
																		%
%!!!!!!!!!!!!!!!!!!!!!!!!!!!!!!!!!!!!!!!!!!!!!!!!!!!!!!!!!!!!!!!!!!!!!%%%
%%%%%%%%%%%%%%%%%%%%%%%%%%%%%%%%%%%%%%%%%%%%%%%%%%%%%%%%%%%%%%%%%%%%%%%%%
% Configuración automática según el identificador elegido
\input{include/configuraciontitulacion} 

% Información añadida a las propiedades del archivo PDF.
\hypersetup{
	pdftoolbar=false,        % show Acrobat’s toolbar?
	pdfmenubar=false,        % show Acrobat’s menu?
	pdffitwindow=true,     % window fit to page when opened
	pdfauthor = {\miNombre},
	pdftitle = {\titulo},
	pdfkeywords={Sistema Informático} {Scrum} {Silencio Administrativo} {React} {React Native} {Firebase}, % list of keywords
	pdfnewwindow=true,      % links in new window
	colorlinks=true,       % false: boxed links; true: colored links
	linkcolor=black,          % color of internal links (change box color with linkbordercolor)
	citecolor=black,        % color of links to bibliography
	filecolor=magenta,      % color of file links
	urlcolor=cyan           % color of external links
}

%%
% Archivo de acrónimos
%%
\makeglossaries % Genera la base de datos de acrónimos
% Lista de acrónimos (se ordenan por orden alfabético automáticamente)

% La forma de definir un acrónimo es la siguiente:
% \newacronyn{id}{siglas}{descripción}
% Donde:
% 	'id' es como vas a llamarlo desde el documento.
%	'siglas' son las siglas del acrónimo.
%	'descripción' es el texto que representan las siglas.
%
% Para usarlo en el documento tienes 4 formas:
% \gls{id} - Añade el acrónimo en su forma larga y con las siglas si es la primera vez que se utiliza, el resto de veces solo añade las siglas. (No utilices este en títulos de capítulos o secciones).
% \glsentryshort{id} - Añade solo las siglas de la id
% \glsentrylong{id} - Añade solo la descripción de la id
% \glsentryfull{id} - Añade tanto  la descripción como las siglas

%\newacronym{tfg}{TFG}{Trabajo Final de Grado}
%\newacronym{tfm}{TFM}{Trabajo Final de Máster}

\newacronym{iso}{ISO}{La Organización Internacional de Normalización}
\newacronym{ui}{UI}{User Interface}
\newacronym{tupa}{TUPA}{Texto Único de Procedimientos Administrativos}
\newacronym{uml}{UML}{Lenguaje Unificado de Modelado}

% A las descripciones se les agrega el punto final automaticamente

% ACRONIMOS
\newglossaryentry{javascript}{
	name=JavaScript,
	description={Lenguaje de programación orientado a objetos, diseñado para el desarrollo de aplicaciones cliente/servidor a través de Internet}
}

\newglossaryentry{nodejs}{
	name=NodeJS,
	description={Entorno de Ejecución Multiplataforma que permite ejecutar \gls{javascript} en navegadores, servidores, PC's, smartphones o en cualquier sitio donde se consiga portar un motor V8 de Google}
}

\newglossaryentry{framework}{
	name=Framework,
	description={Es un conjunto integrado de herramientas que facilitan un desarrollo software}
} % Archivo que contiene los acrónimos

%%%%%%%%%%%%%%%%%%%%%%%% 
% INICIO DEL DOCUMENTO
% A partir de aquí debes empezar a realizar tu tesis
%%%%%%%%%%%%%%%%%%%%%%%%
\begin{document}

% Números romanos hasta el mainmatter.
\frontmatter

% PORTADA
\input{include/portada/portada_unprg} % Portada Unprg
\if\OptimizaTikZ 1
\tikzexternaldisable % Desactiva la exportación com figura
\fi 

% Tamaño por defecto de la fuente de texto para:
\def\FuenteTamano{55pt}	% Tamaño para el título del trabajo
\def\interlinportada{5.0} % Interlineado por defecto para el título
\def\TamTrabajo{20pt} 	% Tamaño para el tipo de trabajo (grado o máster)
\def\TamTrabajoIn{17pt} 	% Tamaño para el salto de línea después de tipo de trabajo
\def\TamOtros{14pt} 	% Tamaño para datos personales y fecha
\def\TamOtrosIn{14pt} 	% Tamaño para los saltos de línea en la info personal
\def\TamOtrosJP{11pt} 	% Tamaño para los saltos de línea en la info personal

\begin{titlepage}

% Márgenes de esta pagina modificados
\newgeometry{ignoreall,top=2cm,bottom=2cm,left=2.4cm,right=2.4cm}

% Offset horizontal para toda la portada
%\setlength{\centeroffset}{0.5\oddsidemargin}
%\addtolength{\centeroffset}{0.5\evensidemargin}
\thispagestyle{empty}

\begin{tikzpicture}[remember picture,overlay]
\node[anchor=north west,inner sep=0pt] at (-0.8,0.7)
{\includegraphics[height=4cm]{\logoUniversidadPortadaOriginal}};
\end{tikzpicture}

\begin{tikzpicture}[remember picture,overlay]
\node[anchor=north west,inner sep=0pt] at (12.8,1.2)
{\includegraphics[height=4cm]{\logoFacultadOriginal}};
\end{tikzpicture}

\centering

\begin{adjustwidth}{6em}{6em}
	\centering
	\begin{spacing}{1.5}
		{\fontsize{\TamTrabajoIn}{18pt} \textbf{\MakeUppercase{\miUniversidadTwoLine}}}\\[0.5cm]
	\end{spacing}
\end{adjustwidth}
%\vspace{0.5cm}
\begin{adjustwidth}{4em}{4em}
	\centering
	\begin{spacing}{1.2}
		{\fontsize{\TamOtros}{0pt} \textbf{\MakeUppercase{\miFacultad}}}\\[0.5cm]
		{\fontsize{\TamOtros}{0pt} \textbf{\MakeUppercase{\miEscuela}}}\\[0cm]
	\end{spacing}
\end{adjustwidth}
{\fontsize{\TamOtrosIn}{0pt} \textbf{Tesis}}\\[0.5cm]

\begin{spacing}{1.5}
	\centering
	{\fontsize{\TamOtrosIn}{0pt} \textbf{``\titulo, \subtitulo.''}}\\[1cm]
\end{spacing}

{\fontsize{\TamOtrosIn}{0pt} \textbf{Para optener el Título Profesional de:}}
\\[0.5cm]
{\fontsize{\TamOtrosIn}{0pt} \textbf{{Ingeniero de Sistemas}}}
\\[1cm]
{\fontsize{\TamOtrosJP}{0pt} \textbf{Aprobado por los Miembros del Jurado}}
\\[0cm]
\vspace{2cm}
\begin{multicols}{2}
	\centering
	\vspace{2cm}
	Mg. Ing. Apellidos, Nombres Jurado 1\\
	Presidente\\
	\vspace{2cm}
	Ing. Apellidos, Nombres Jurado 3\\
	Vocal\\
	\columnbreak
	\vspace{2cm}
	Ing. Apellidos, Nombres Jurado 2\\
	Secretario\\
	\vspace{2cm}
	Mg. Ing. Apellidos, Nombres Asesor\\
	Asesor\\
\end{multicols}
\vspace{2cm}
Apellidos, Nombres Autor\\
Autor\\
\end{titlepage}

\if\OptimizaTikZ 1
\tikzexternalenable % Reactiva la exportación como figura
\fi
% A partir de aquí aplica los márgenes establecidos en configuracioninicial.tex
\restoregeometry % Portada Firmas

%%%%% PREAMBULO
% Incluye el .tex que contiene el preámbulo, agradecimientos y dedicatorias.
% Aquí va la dedicatoria si la hubiese. Si no, comentar la(s) linea(s) siguientes
\chapter*{Dedicatoria}
\label{empty}
\setlength{\leftmargin}{0.5\textwidth}
\setlength{\parsep}{0cm}
\addtolength{\topsep}{0.5cm}
\begin{flushright}
	\small\em{
		Lorem Ipsum is simply dummy text of the printing\\
		and typesetting industry. Lorem Ipsum has been the\\
		industry's standard dummy text ever since the 1500s\\~\\
		Nombres y Apellidos del Autor.\\~\\~\\~\\~\\
	}
\end{flushright}

%\cleardoublepage %salta a nueva página impar

\chapter*{Agradecimientos}
\label{agradecimientos}

\thispagestyle{empty}
\vspace{1cm}
\par \lipsum[1-1]

%\cleardoublepage %salta a nueva página impar
 
\chapter*{Resumen}
\label{resumen}

\par \lipsum[1-1].\\

\par Palabras claves: Lorem, Ipsum, dummy, industry.

\chapter*{Abstract}
\label{abstract}
\par \lipsum[1-1].\\

\par Keywords: Lorem, Ipsum, dummy, industry.

% Incluye después del archivo anterior el indice y lista de figuras, tablas y códigos.
\tableofcontents	% Índice
\listoffigures		% Índice de figuras
\listoftables		% Índice de tablas
%\lstlistoflistings	% Índice de códigos
\listofappendices	% Índice de Apéndices
% Inicia la numeración habitual.
\mainmatter
% Padding en las Tablas
{\renewcommand{\arraystretch}{1.4}
%%%%
% CONTENIDO. CAPÍTULOS DEL TRABAJO - Añade o elimina según tus necesidades
%%%%
\chapter{Introducción}
\label{introduccion}

\par \lipsum[2-5].\\

\par Ejemplo de acrónimo completo: \enquote{\acrfull{tupa}}.\\	% Plantilla: Se muestran contenidos
\input{capitulos/MarcoTeorico} % NUEVO
\chapter{Marco Metodológico}
\label{metodologico}

\section{Hipótesis de Investigación}

\lipsum[4-4]

\section{Diseño de Investigación}

\lipsum[3-3]

\section{Operacionalización de variables}

\lipsum[2-2]

\subsection{Definición Conceptual de las Variables}

\lipsum[1-1]

\subsection{Definición Operacional de las Variables}

\lipsum[2-2]

\subsection{Indicadores Experimentales}

\lipsum[3-3]

\section{Población y Muestra}

\lipsum[1-1]

\subsection{Técnicas e instrumentos de recolección de datos}

\subsubsection{Técnicas de recolección de datos}

\par Las técnicas para la recolección de datos serán las siguientes:

\paragraph{Técnica Uno}

\par \lipsum[1-1]

\subsection{Instrumentos de recolección de ratos}

\par \lipsum[1-1] % NUEVO
\chapter{Resultados}
\label{resultados}
\lipsum[1-1]

\section{Sección Uno}

\par \lipsum[1-1]

\subsection{Subsección Uno.Uno}

\lipsum[3-3]

\subsubsection{Subsubsección Uno.Uno.Uno}

\lipsum[2-2]

\paragraph{\lipsum[4-4]}

\section{Sección Dos}

\par \lipsum[2-2]
 % NUEVO
\chapter{Discusión de Resultados}
\label{discusion}

\lipsum[1-1]

\section{Contrastación de la Hipótesis de Investigación}
\lipsum[1-1]

\subsection{Subsección Uno}

\lipsum[5-5] % NUEVO
\chapter{Conclusiones}
\label{conclusiones}

\lipsum[1-1]

\begin{itemize}
	\item \lipsum[1-1]
	\item \lipsum[2-2]
\end{itemize} % NUEVO
\chapter{Recomendaciones}
\label{recomendaciones}
\par \lipsum[1-1]

\begin{itemize}
	\item \lipsum[1-1]
	\item \lipsum[2-2]
\end{itemize} % NUEVO
%%%%
% CONTENIDO. BIBLIOGRAFÍA.
%%%%
\nocite{*} %incluye TODOS los documentos de la base de datos bibliográfica sean o no citados en el texto
\bibliography{bibliografia/bibliografia} % Archivo que contiene la bibliografía
\bibliographystyle{apacite}
%%%%
% CONTENIDO. LISTA DE ACRÓNIMOS. Comenta las líneas si no lo deseas incluir.
%%%%
% Incluye el listado de acrónimos utilizados en el trabajo. 
\printglossary[style=modsuper,type=\acronymtype,title={Lista de Acrónimos y Abreviaturas}]
\printglossary[style=modsuper, title={Glosario de Términos}]
% Añade el resto de acrónimos si así se desea. Si no elimina el comando siguiente
\glsaddallunused 
%%%%
% CONTENIDO. Anexos - Añade o elimina según tus necesidades
%%%%
\appendix % Inicio de los apéndices
\chapter{Colecciones de Datos}
\label{ape:collections}

\par \lipsum[2-4]
\backmatter


\end{document}
